\documentclass[letterpaper,10pt]{book}
%\usepackage{biblatex}
\usepackage[hidelinks]{hyperref}
\usepackage[utf8]{inputenc}
\usepackage{makeidx,multicol}
\makeindex
\usepackage{minted}
\usepackage{xcolor}
\usepackage{textcomp} % for angle brackets....
\newcommand{\deemph}[1]{{\color{black!40}#1}}

\title{Epode: Reference Manual}
\author{Martin Jay McKee\\martinjaymckee@gmail.com}
%\date{}

\usemintedstyle{friendly}
\newminted[epodecode]{cpp}{mathescape,
               linenos,
               numbersep=5pt,
               framesep=2mm}

\newmintinline[epodeinline]{cpp}{}

\newcommand{\srcas}[1]{\textbf{include} \textlangle{}\textit{#1}\textrangle{}}
\newcommand{\epode}[0]{\textbf{Epode}}

\begin{document}
\maketitle 
\tableofcontents

\part{Just the Basics}
  \chapter{Getting Started}
    \section{Overview}
      Epode is a \deemph{library} for numeric integration of first-order systems of ordinary differential equations for the solution of inital value problems.  That is, given a system of equations, an integration range, and an intial value of the form,

      \begin{equation}
	\dot{\vec{x}} = f(t, \vec{x}), t = [t_{0}, t_{1}], \vec{x}_{0} = ...  
      \end{equation}

      Epode can calculate an estimate of the final value of the system variables $\vec{x}$.  Some problems of this type can be solved analytically.  The vast majority of this type are, however, analytically intractible and, as such require some form of numeric solution.  This is where an ode solver like \epode{} comes in.

      \subsection{What's with the name anyway?}
	\epode{} is a C++14 compatible library for integration of ordinary differential equations (odes). The connection is simple, an ode is a lyric poem written to address a particular subject.  This library was written to address the need for a clean, easy to use and efficient library for the solution of ordinary differental equations.  The final link (as simply naming the library "ode" would be too boring) is that the word epode is virtually synonomous to ode and refers to a specific form of lyric poem typified by couplets composed of lines with, typically, varying length (not that that matters in any way to this library).  The Epode library is an ode to the new features of C++ 11/14 which make such a library much easier to create and use.  It doesn't hurt that it's a short name and contains the letters 'o', 'd', 'e', either!      
      

\index{All}\index{Apple}\index{Ball}\index{Cherry}\index{ODE}\index{Method}\index{Zoo}

    \section{Installation}

    \section{A Boring Example}
  \begin{listing}
    \label{lst:basic}
    \caption{This is how to create and run a solver}  
    \begin{epodecode}
      using namespace epode;
      using solver_t = epode::integrator::Euler<double, 3>;
      using state_t = typename solver_t::state_t;
      auto f = [](auto v, auto y){
	return {
	  y(0) + v*y(1),
	  -y(1),
	  y(0) * y(2)
	}
      } -> state_t;
      auto y0 = state_t{1, 2, 3};
      auto solver = Solver(0.01); // Step integration variable by 0.01
      solver(f, 0, {0.5, 1, 1.5, 2}, y0);    
    \end{epodecode}
  \end{listing}

  \chapter{User API}
    TALK ABOUT THE PARTS OF THE LIBRARY... INTRODUCE THE SECTIONS

    \section{The ``solve'' Function}
      The function \epodeinline{solve(...)} is syntactic sugar for a number of steps to develop an Epode solver and run it.  Still, for simple uses of \epode{}, it should be fully sufficient.
      
    \section{Integrator Engine}
      The integrator engine is the object that contains the implementation of the method ``stepping'', as well as storage of the integration results.
      
    \section{Methods}
      \epode{} provides a number of integration methods, all the way from the $1^{st}$ order Euler's method up to the ?? method with adaptive stepping.  These methods can be selected explicitly by the user either when using the basic ``solve'' function or when constructing an integration flow directly.
      
    \section{Trigger Objects}
      In the \epode{} library, Several functions of the integration are controlled by - so called - trigger objects.

\part{The Library in Depth}
  \chapter{Solver Methods}
    \section{Solver API}
    
      \subsection{End/Save Syntax}
	There are multiple ways to denote the points where a solver either ends integration or saves the calculated values
	
      \subsection{Save Transformers}
	ADD A WAY TO ``TRANSFORM'' THE STATE BEFORE IT IS OUTPUT.  THIS SHOULD ALLOW FOR CHANGING THE SIZE OF THE STATE AS WELL AS DOING SOME NUMBER OF OPERATIONS ON THE STATE BEFORE SAVING IT.
	
	SYNTAX -- transform(stats, $y_{n}$, $\dot{}v_{n}$, $v_{n}$, $y_{n-1}$)
	
      \subsection{Results Type}
      IT WOULD BE NICE TO MAKE IT POSSIBLE FOR THE RESULTS PACKAGE TO BE OF FIXED SIZE (I.E. NOT A STD::VECTOR) SO THAT DYNAMIC ALLOCATIONS CAN BE AVOIDED).  THIS WILL REQUIRE DEFINING THE RETURN TYPE OF THE INTEGRATOR OPERATOR () BASED ON THE TYPE OF THE STORE TRIGGER.  ONCE THAT WORKS, AS LONG AS THE RETURN TYPES ALL HAVE A SIMILAR API, IT COULD BE A SINGLE VALUE, ARRAY OR VECTOR.
      
      
    \section{Explicit Methods}
    
      EACH OF THE METHODS SHOULD HAVE A DESCRIPTION OF THE METHOD, THE BUTCHERS TABLEAU, AND ANY REFERENCES.  ADDITIONALLY, THE FILE THAT CONTAINS THE IMPLEMENTATION SHOULD BE STATED.
      
      \subsection{Forward Euler}   
	\[
	  \begin{array}{c|c}
	    0 & 0\\
	    \hline
	    & 1
	  \end{array}
	\]
	
	Available with: \srcas{Euler}
	
      \subsection{Generic 2nd-Order Runge-Kutta}
	IS THERE A GOOD WAY TO MAKE SUCH A PARAMETRIC METHOD?? THIS COULD USE A VALUE PASSED TO THE METHOD'S CONSTRUCTOR
	
	TODO: WRITE THE GENERIC 2ND ORDER AND THEN IMPLEMENT HEUN'S, EXPLICIT MIDPOINT AND RALSTON'S USING IT.
	
	\[
	  \begin{array}{c|cc}
	    0 & & \\
	    \eta & \eta & \\
	    \hline
	    & 1-1/2\eta & 1/2\eta
	  \end{array}
	\]

	Available with: \srcas{RK2}
	    
      \subsection{Heun}
	\[
	  \begin{array}{c|cc}
	    0 & & \\
	    1 & 1 & \\
	    \hline
	    & 1/2 & 1/2
	  \end{array}
	\]

	Available with: \srcas{RK2}
	
      \subsection{Midpoint}
	\[
	  \begin{array}{c|cc}
	    0 & & \\
	    1/2 & 1/2 & \\
	    \hline
	    & 0 & 1
	  \end{array}
	\]    

	Available with: \srcas{RK2}
	
      \subsection{Ralston's}
	\[
	  \begin{array}{c|cc}
	    0 & & \\
	    2/3 & 2/3 & \\
	    \hline
	    & 1/4 & 3/4
	  \end{array}
	\]      

	Available with: \srcas{RK2}

      \subsection{RKF1(2)}
	\[
	  \begin{array}{c|ccc}
	    0 & & &\\
	    1/2 & 1/2 & &\\
	    1 & 1/256 & 255/256 &\\	    
	    \hline
	    & 1/256 & 255/256 & 0\\
	    & 1/512 & 255/256 & 1/512\\
	  \end{array}
	\]    
    
	Available with: \srcas{RKF12}
    
      \subsection{Euler/Heun}
    
    	Available with: \srcas{Euler}

      \subsection{RK3}
	Available with: \srcas{RK3}
    
    
      \subsection{Bogacki–Shampine 3(2)}
	\[
	  \begin{array}{c|cccc}
	    0 & & & &\\
	    1/2 & 1/2 & & &\\
	    3/4 & 0 & 3/4 & &\\	    
	    1 & 2/9 & 1/3 & 4/9 &\\	    
	    \hline
	    & 2/9 & 1/3 & 4/9 & 0\\
	    & 7/24 & 1/4 & 1/3 & 1/8\\
	  \end{array}
	\]    
    	Available with: \srcas{BS32}

    
      \subsection{RK4}
    	Available with: \srcas{RK4}

    
      \subsection{RKF4(5)}
	Available with: \srcas{RKF45}
    

      \subsection{Butcher's 5th}
	Available with: \srcas{Butchers5}
    
    
    %\section{Implicit Methods}

  \chapter{The Integrator Process}
  \index{Method}\index{ODE}
    
    \section{Triggers}
    
    \section{Logging}
    WHILE LOGGING WOULD BE A NICE FEATURE, IT CURRENTLY ISN'T IMPLEMENTED.  FIGURE OUT HOW TO DO THAT AND THEN ADD IT

  \chapter{Extension API}

  
\part{Usage Examples}

  \chapter{Toy Problems}
    SHOULD THE TOY PROBLEMS ALL BE SOLVABLE ANALYTICALLY?
    
    \section{Capacitor Discharge}
      THIS IS EASY TO SOLVE ANALYTICALLY -- USE END VOLTAGE TRIGGER??? (CUSTOM TRIGGER)

    \section{A Random Complex Valued IVP}
      JUST SOMETHING SIMPLE LIKE $\dot{z} = (z - t)^{2}$

    \section{Van der Pol Oscillator}
      THIS CAN SHOW A SIMPLE TWO VARIABLE, MODERATELY STIFF SYSTEM

  \chapter{Physical Simulation}
    \section{Ballistic Modeling}
      Some of the earliest mechanical and electronic computers were created to assist in the calculation of ballistics.  Why not do the same here?
      
      IT WOULD BE NICE TO DO A BALLISTIC MODEL THAT ALLOWS FOR TESTING WITH AND WITHOUT DRAG.  IS THERE A WAY TO DO THIS IN THE ODE SOLVER, OR IS THAT PART OF THE SYSTEM DESCRIPTION?
      
      BASE THIS ON \cite{wade2011going}.

    \section{Pendulum}
      USE THE NON-LINEAR SECOND-ORDER EQUATION... AFTER CONVERSION TO FIRST-ORDER FORM, SOLVE SYSTEM
      $m\ddot{\theta} + \lambda\dot{\theta} + \frac{m g}{L}sin(\theta) = 0$
      AS FOUND IN, ``https://nrich.maths.org/content/id/6478/Paul-not so simple pendulum 2.pdf''

    \section{Predator/Prey}
  
  \chapter{Chaotic Attractors}
    \section{Lorenz System}
    \section{Rossler Attractor}
    \section{Chua's Circuit}
      Chua's circuit was intended to show that a physical electronic circuit could demonstrate chaotic behavior, as documented by Chua himself in~\cite{chua1992genesis}.  The circuit, like the Lorenz system and the Rossler system, is described by an ODE with three state variables.  For this example, we shall use the implementation of the circuit described in~\cite{kennedy1992robust}.


\part{Appendicies}
  \chapter{Basics of ODEs}
  
  \chapter{Analysis of Numeric Methods}
  
  \chapter{Troubleshooting}
    LIST POSSIBLE ERRORS WITH THEIR CAUSE AND SOLUTION, ETC.
    
\bibliography{epodemanualreferences}
\bibliographystyle{unsrt}
  
\printindex

\end{document}
